\documentclass[a4paper]{article}

\usepackage{longtable,booktabs}
\def\tightlist{}

\usepackage{hyperref}

\usepackage[margin=1cm]{geometry}

\setcounter{secnumdepth}{0}

\pagenumbering{gobble}

\title{\bfseries\Huge Chris Bamford}

\author{\textbf{Email:}
\href{mailto:chrisbam4d@gmail.com}{\nolinkurl{chrisbam4d@gmail.com}} -
\textbf{Mobile:} \href{tel:+447853262719}{+44-7853-262-719}
\textbf{Twitter:} \href{https://twitter.com/bam4d}{@Bam4d}
\textbf{Github:} \href{https://github.com/Bam4d}{Bam4d}}

\begin{document}

\maketitle

\section{Education}\label{education}

\begin{itemize}
\tightlist
\item 
  \textbf{PhD - Artificial Intelligence in Gaming, Queen Mary University (Expected August 2022)}
\item
  \textbf{First Class Hons MEng in Applied Cybernetics, University of
  Reading.}
\item
  3 A-levels (3 A), Dr Challoner's Grammar School, United Kingdom.
\item
  12 GCSEs (6 A*, 4 A), Dr Challoner's Grammar School, United Kingdom.
\end{itemize}

\section{Recent Publications}\label{publications}

\begin{description}
\tightlist
\item[2021]
\textbf{Griddly - A platform for AI research in games}

Software Impacts 8, 100066 + AAAI Workshop on reinforcement Learning

URL: \url{https://arxiv.org/abs/2011.06363}
\item[2021]
\textbf{Generalising Discrete Action Spaces with Conditional Action Trees}

2021 IEEE Conference on Games (CoG), 1-8

URL: \url{https://arxiv.org/abs/2104.07294}
\item[2021]
\textbf{Gym uRTS - Toward Affordable Full Game Real-time Strategy Games Research with Deep Reinforcement Learning}

2021 IEEE Conference on Games (CoG), 1-8

URL: \url{https://arxiv.org/abs/2105.13807}
\item[2020]
\textbf{Neural Game Engine: Accurate learning of generalizable forward models from pixels.}

2020 IEEE Conference on Games (CoG), 81-88

URL: \url{https://arxiv.org/abs/2003.10520}
\end{description}

\section{Professional Experience}\label{professional-experience}

\begin{description}

\item[2022 March-Present]
\textbf{Internship (continuation) MetaAI London}

During my internship at MetaAI I made a POC of a level building and editing tool for the MiniHack learning environment. I was invited back as a contractor to continue this work.

\item[2021 June-September]
\textbf{Internship MetaAI London}

For my internship with MetaAI, I worked on algorithms to improve training in IMPALA algorithm under data augmentation. I identified several problematic theoretical implications that negatively affect training and provided and evaluated methods to mitigate these effects. These methods centered around consistency regularization and geometric deep learning.

\item[2018 Q4-present]
\textbf{PhD - Artificial Intelligence in Gaming, Queen Mary University}

My research interests are primarily in Deep Reinforcement leaning
particularly using model-based planning as a method of improving sample
efficiency and model accuracy. Model-based methods are a particuarly
interesting area of research as they allow artificial agents to use
tools which have psychological parallels such as imagination, curiosity
and empowerment. I believe that these tools are a step towards general
artificial intelligence.
\item[2018 Q3]
\textbf{Founder - embod.ai}

embod.ai is a start-up where researchers, students and hobbiests can
build and observe AI agents in Massively Multiplayer Online
environments. The goal of embod.ai is to make AI based gaming and
competitions a mainstream e-sport. Beta release:
\url{https://medium.com/@chris.bamford/embod-ai-come-join-the-beta-efcdbc85f524}

\item[2018 Q2]
\textbf{EF10LDN Cohort Member - Entrepreneur First}

Entrepreneur First is the UK's most successful startup accellerator
programme. Entrepreneur First gathers up to 100 highly talented
individuals in two cohorts a year and helps them to build globally
important companies.
\item[2015-2017]
\textbf{Lead Research and Development engineer at import.io}

import.io specializes in making it easy for non-technical users to
gather and structure data from the web. I have worked at import.io since
it was founded in 2012. I have worked mainly on designing and creating
innovative technologies to generate a leading edge over any competitors.
I have worked mainly in Java when building infrastructure and backend
projects, and Python when doing Machine Learning.

Some of the projects of which I have had significant input and I have
enjoyed working on:

\begin{itemize}
\tightlist
\item
  \textbf{Bees} - import.io's highly distributed and scalable querying
  and messaging platform. Bees handles many challenges that are faced
  when building a distributed web extraction system, such as variable
  and unknown network latency, asynchronous IO across multiple servers
  and thousands of endpoints, pipelining and sorting of sparse data sets
  and collation of data from multiple sources.
\item
  \textbf{Label Suggestion} - A statistical model that would return the
  ``most likely next label'' that a user would chose for a data column
  given the previous data columns the user had inputted. Label
  suggestion used some highly optimized BLAS libraries to quickly infer
  probabilities over a large data set of manually labelled training
  data.
\item
  \textbf{Deep Learning} - Research projects with deep learning and web
  data extraction. For example I worked on several page classification
  approaches using Recurrent Neural networks and Recursive Neural
  networks. Additionally I worked on data point recognition where i
  worked to develop algorithms that would classify where there were
  datapoints in a webpage and then classify what label those data points
  should be given. The research was performed using a few different deep
  learning libraries; deeplearning4j, neon, keras (experimenting with
  tensorflow and theano).
\end{itemize}
\item[2012-2015]
\textbf{Software Engineer at import.io}

Before the machine learning and algorithm team was created at import.io,
We were a small team of 3-5 engineers. At this time I mainly worked on
the main Java codebase.

\end{description}

\section{Notable Projects}\label{notable-personal-projects}

\begin{description}
\tightlist

\item[\textbf{Griddly}]
Griddly is an open-source project aimed to be a all-encompassing platform for grid-world based research. 
    Griddly provides a highly optimized game state and rendering engine with a flexible high-level interface for configuring environments. 
    Not only does Griddly offer simple interfaces for single, multi-player and RTS games, but also multiple methods of rendering, configurable partial observability and interfaces for procedural content generation.

\end{description}

\section{Open Source Contributions}\label{open-source-contributions}

\begin{description}
\tightlist
\item[\textbf{Entity Neural Networks}]
This is an ongoing project to provide implemenations and baselines for transformer models in reinforcement learning environments. More specifically the Griddly project natively provides interfaces for these types of models.

Commits: https://github.com/entity-neural-network/incubator/commits?author=Bam4d
\item[\textbf{RLLib}]
Have made several contributions to RLLib, specifically around threading and memory usage in their implementation of the distributed IMPALA algorithm.

Commits: https://github.com/ray-project/ray/commits?author=bam4d

\end{description}

\section{Key Skills}\label{key-skills}

\begin{description}
\tightlist
\item[C/C++]
The majority of Griddly is written in C/C++ with bindings to Python, Julia and Javascript. 
\item[Python]
I have worked with python in many different types of projects, from
research with deep learning libraries such as pytorch and tensorflow, pytorch; To building REST API services using
libraries such as Flask when productionizing machine learning models.
\item[Java]
The majority of my work at import.io was using Java. I have used Spring, SpringBoot,
Hazelcast, Jedis, Jackson, Lettuce, Jetty 8/9, mockito, cucumber,
gherkin and Google libraries such as Guava, Lombok and Futures.
\item[Javascript/Typescript]
I have mande several demos and smaller projects using React, such as simple level designers for Griddly and MiniHack learning environments.
\item[Amazon AWS]
I have used a large number of features of amazon aws while at import.io,
embod.ai and in some personal projects. I am comfortable with using
Lambda, CloudFormation, ECR, ECS, ElasticBeanstalk, Route53, S3, EC2,
Redshift, AuroraDB and ElastiCache
\end{description}

\end{document}