\documentclass[a4paper]{article}

\usepackage{longtable,booktabs}
\def\tightlist{}

\usepackage{hyperref}

\usepackage[margin=1cm]{geometry}

\setcounter{secnumdepth}{0}

\pagenumbering{gobble}

\title{\bfseries\Huge Chris Bamford}

\author{\textbf{Email:}
\href{mailto:chrisbam4d@gmail.com}{\nolinkurl{chrisbam4d@gmail.com}} -
\textbf{Mobile:} \href{tel:+447853262719}{+44-7853-262-719}}

\begin{document}

\maketitle

\section{Education}\label{education}

\begin{itemize}
\tightlist
\item
  \textbf{First Class Hons MEng in Applied Cybernetics, University of
  Reading.}
\item
  3 A-levels (3 A), Dr Challoner's Grammar School, United Kingdom.
\item
  12 GCSEs (6 A*, 4 A), Dr Challoner's Grammar School, United Kingdom.
\end{itemize}

\section{Publications}\label{publications}

\begin{description}
\tightlist
\item[2012]
\textbf{Modular non-computational-connectionist-hybrid neural network
approach to robotic systems.}

Paladyn. Journal of Behavioral Robotics. February 6, 2012

URL: \url{https://link.springer.com/article/10.2478/s13230-012-0003-6}
\item[2010]
\textbf{Cavalcade neural network for mobile robot.}

Cybernetic Intelligent Systems, 7th IEEE International Conference,
September 1-2,

URL: \url{http://ieeexplore.ieee.org/document/5898087/}
\end{description}

\section{Professional Experience}\label{professional-experience}

\begin{description}
\item[2015-today]
\textbf{Lead Research and Development engineer at import.io}

import.io is a start-up of 10-50 employees that specializes in making it
easy for non-technical users to gather and structure data from the web.
I have worked at import.io since it was founded in 2012. I have worked
mainly on designing and creating innovative technologies to generate a
leading edge over any competitors. I have worked mainly in Java when
building infrastructure and backend projects, and Python when doing
Machine Learning.

Some of the projects of which I have had significant input and I have
enjoyed working on:

\begin{itemize}
\tightlist
\item
  \textbf{Bees} - import.io's highly distributed and scalable querying
  and messaging platform. Bees handles many challenges that are faced
  when building a distributed web extraction system, such as variable
  and unknown network latency, asynchronous IO across multiple servers
  and thousands of endpoints, pipelining and sorting of sparse data sets
  and collation of data from multiple sources. Bees also had to record
  and supply metrics for the business such as success rates, per-user
  statistics and billing information. Uses technologies such as
  \emph{Spring, Redis, S3, EC2, ElasticBeanstalk, CloudFormation and
  Docker}
\item
  \textbf{Label Suggestion} - A statistical model that would return the
  ``most likely next label'' that a user would chose for a data column
  given the previous data columns the user had inputted. Label
  suggestion used some highly optimized BLAS libraries to quickly infer
  probabilities over a large data set of manually labelled training
  data.
\item
  \textbf{Deep Learning} - Research projects with deep learning and web
  data extraction. For example I worked on several page classification
  approaches using Recurrent Neural networks and Recursive Neural
  networks. Additionally I worked on data point recognition where i
  worked to develop algorithms that would classify where there were
  datapoints in a webpage and then classify what label those data points
  should be given. The research was performed using a few different deep
  learning libraries; deeplearning4j, neon, keras (experimenting with
  tensorflow and theano).
\end{itemize}
\item[2012-2015]
\textbf{Software Engineer at import.io}

Before the machine learning and algorithm team was created at import.io,
We were a small team of 3-5 engineers. At this time I mainly worked on
the main Java codebase.

Some of the projects of which I have had significant input and I have
enjoyed working on:

\begin{itemize}
\tightlist
\item
  \textbf{``Magic''} - An algorithm that finds and extracts lists of
  records in a webpage. This product made use of several different
  matching and allocation techniques such as Jaccard distance,
  Hirschberg's algorithm and Munkres(Hungarian) algorithm. The average
  time to detect and extract the top ``most likely'' tables was less
  than a second. \emph{The success of this project lead me to become the
  Lead Research and Development engineer.}
\item
  \textbf{Object Store} - A distributed platform for storing and
  retrieving data objects from an underlying database, originally a
  Percona SQL database was stored, with binary attachment storage on
  amazon S3, Later this was migrated to use amazon auroraDB.
\item
  \textbf{Job Service} - The precursor system to ``bees'', handled 200
  user web extraction requests a second, distributing the requests
  around a cluster of 50 servers at full scale. Used Hazelcast for
  distributed map and queueing system.
\end{itemize}
\item[2010-2012]
\textbf{CTO at TouchGo Technology}

\begin{itemize}
\tightlist
\item
  Responsible for the development and integration of all technologies
  involved in the vending machine which won the Auto Vending Innovation
  Award 2010 for Machine Intelligence.
\end{itemize}
\end{description}

\section{Open Source Contributions}\label{open-source-contributions}

\begin{description}
\tightlist
\item[\textbf{deeplearning4j/nd4j}]
Helped to write one of the first versions of GPU interfacing for the
deep learning 4 java library developed by skymind.io

Commits:
\url{https://github.com/deeplearning4j/nd4j/commits?author=Bam4d}
\item[\textbf{Neon}]
Added a statistics callback which contains precision and recall metrics
for multi-label classification problems.

Commits:
\url{https://github.com/NervanaSystems/neon/commits?author=Bam4d}
\end{description}

\section{Personal Projects}\label{personal-projects}

\begin{description}
\tightlist
\item[\textbf{BLIF}]
The successor to Neuretix, a spiking neural network library that takes
inspiration from gradient descent based learning techniques in deep
learning and applies it to the same leaky integrate and fire models that
neuretix uses.

This project is a work in progress and supercedes the Neuretix project.

Github: \url{https://github.com/Bam4d/BLIF}
\item[\textbf{Roxxy}]
A headless rendering engine that wraps the Chromium embedded framework
with Facebook's proxygen libraries. This allows a browser to act as a
server, requesting full rendered web pages using an API. It also allows
multiple web requests to be made in parallel.

Github: \url{https://github.com/Bam4d/Roxxy}
\item[\textbf{Neuretix}]
A spiking neural network implementation using the models I developed as
part of my two publications. Neuretix uses a novel ring-buffer data
structure to queue spikes. This allows computational complexity of
forward pass updates reduce from `O(n\^{}2)' to `O(n)', where `n' is
proportional to the number of neurons firing at a single time step.

Github: \url{https://github.com/Bam4d/Neuretix}
\item[\textbf{Neural Stacks (Keras deep learning library)}]
Although I did not contribute this implementation back to the Keras
group, I enjoyed implementing the differentiable stack algorithm and
seeing it work.

Github: \url{https://github.com/Bam4d/keras/tree/neural-stack}
\end{description}

\section{Key Skills}\label{key-skills}

\begin{description}
\tightlist
\item[Python]
I have worked with python in many different types of projects, from
research with deep learning libraries such as neon, keras, theano,
tensorflow and scikitlearn; To building REST API services using
libraries such as Flask when productionizing machine learning models.
\item[Java]
I am proficient in Java 8 paradigms such as futures, streams, lambdas
and fluent APIs. I have also been following the release notes for java 9
and have started incorporating Reactive streams (Spring Project Reactor)
into the code that I am working on. I have used Spring, SpringBoot,
Hazelcast, Jedis, Jackson, Lettuce, Jetty 8/9, mockito, cucumber,
gherkin and Google libraries such as Guava, Lombok and Futures.
\item[Amazon AWS]
I have used a large number of features of amazon aws while at import.io
and in some personal projects. I am comfortable with using Lambda,
CloudFormation, ECR, ECS, ElasticBeanstalk, Route53, S3, EC2, Redshift,
AuroraDB and ElastiCache
\item[C/C++]
Neuretix, Roxxy and a few other of my personal projects and open source
contributions not mentioned here have been in C/C++. C++ is probably my
favourite programming language. It allows you to squeeze performance out
of many algorithms and programming methods where other programming
languages are too restrictive.
\end{description}

\end{document}